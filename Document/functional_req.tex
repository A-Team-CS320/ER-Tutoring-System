    \chapter{Functional Requirements}
%\section{Functional Requirements}
    \begin{section}{Allow Student to Draw a Diagram}
        Description: Student given tools and space to draw an ER diagram \newline
        Primary Actor: Student \newline
        Precondition: Student is logged in and viewing a            question \newline
        Trigger: Student chooses a question to work on \newline
        Success end condition: Student can submit answer            \newline
        Failed end condition: Diagram is not drawn \newline
        \newline
        Steps:
        \begin{enumerate}
            \item{Empty space available to draw diagram}
            \item{Student picks a diagram type (Chen’s/Crow’s Foot)}
            \item{Student picks shapes from toolbox and places in           diagram}
            \item{Student places text inside shapes}
            \item{Student places links between shapes}
            \item{Student can edit or remove anything already placed on the diagram before finalizing}
        \end{enumerate}
        Exceptions:
        \begin{itemize}
            \item{Teacher draws diagram that students must edit             \newline
                *Draw space will not be empty, student will be able to           make changes to present diagram}
        \end{itemize}
    \end{section}

    
    \begin{section}{Submit Answer}
        Description: Student submits answer to be checked and saved by the system \newline
        Primary Actor:Student \newline
        Precondition: Student is ready to submit answer \newline
        Trigger: Pressing the submit button \newline
        Success end condition: Feedback and/or grade is given       \newline
        Failed end condition: Nothing is given back to student      \newline
        \newline
        Steps:
        \begin{enumerate}
            \item{Student submits answer by pressing the submit             button}
            \item{System processes answer and decides validity}
            \item{Outputs feedback and/or grade for student to see}
            \item{Answer is saved into database}
        \end{enumerate}
        Exceptions:
        \begin{itemize}
            \item{Student leaves area blank \newline
        	*Answer is just marked as wrong}
            \item{Author never properly input the correct answer or     feedback \newline
        	*Student is told that there's nothing to report}
        \end{itemize}
    \end{section}

    
    \begin{section}{Allow Teacher to Create    Questions/Feedback}
        Description: Teacher creates a question and adds it to a specific assignment \newline
        Primary Actor: Teacher \newline
        Precondition: Teacher is logged in a specific course. \newline
        Trigger: Teacher selects a specific Homework to create a     question. \newline
        Success end condition: Question is uploaded and the instructor is able to see it. \newline
        Failed end condition: Students don’t see the question       properly. \newline
        \newline
        Steps:
        \begin{enumerate}
            \item{Teacher clicks on a “create” button to create the           question.}
            \item{Teacher select the number of the question that             he/she wants to post for a previously chosen               homework.}
            \item{Teacher writes the database schema of the ER              diagram that is going to be drawn by the student, in         an available textbox.}
            \item{Teacher provides a general feedback in a textbox            provided.}
            \item{Teacher clicks on a “add” button to add the                 question of the homework.}
            \item{Teacher clicks on a “preview” button to see the           students point of view of the entire homework.}
            \item{Teacher clicks on a “submit” button to submit the           entire homework.}
        \end{enumerate}
        Exceptions:
        \begin{itemize}
            \item{Teacher tries to create question but he/she doesn't have Internet connection.}
        \end{itemize}
    \end{section}

    
    \begin{section}{Select Question to Answer}
        Description: Student views a question that he would like to answer \newline
        Primary Actor: Student \newline
        Precondition: Student is logged in and is in an assignment         \newline
        Trigger: The student attempts to select a new question      to attempt to answer \newline
        Success End Condition: Question that the student selects     successfully loads \newline
        Failed End Condition: The question that is selected does     not load \newline
        \newline
        Steps:
        \begin{enumerate}
            \item{Student scrolls through the list of questions in the assignment.}
            \item{Student selects the question they would like to        answer from the list.}
            \item{Webpage for the selected question loads}
            \item{Student can read and answer question}
        \end{enumerate}
        Exceptions: None
    \end{section}
        
    \begin{section}{Student Logs In}
        Description: The student is logging in, in order to access the tutor system using their NetID and password. \newline
        Primary Actor: Student \newline
        Secondary Actor: System \newline
        Precondition: The student is on the login page of the system. \newline
        Trigger: The student wants to log into the tutoring system. \newline
        Successful End Condition: The student successfully logs into the tutoring system. \newline
        Failed End Condition:  The student is unable to log into the system because of an incorrect NetID or password or both \newline
        \newline
        Steps:
        \begin{enumerate}
            \item{The student enters their NetID in the username textbox}
            \item{The student enters their password in the password textbox}
            \item{The student clicks on the LOGIN button to enter the system}
        \end{enumerate}
        Exceptions:
        \begin{itemize}
            \item{If the student does not enter the correct NetID or password the system will 
            not log the student in and a message will appear stating that the NetID or password 
            that they entered is incorrect and to please try again.}
        \end{itemize}
    \end{section}
        
    \begin{section}{Student Views Assignments}
        Description: Once logged in, the student can view available assignments \newline
        Primary Actor: Student \newline
        Precondition: Student is already logged in \newline
        Trigger: Student logs in \newline
        Successful End Condition: Student can view all questions in an assignment \newline
        Failed End Condition: Student is unable to see any questions in the assignment \newline
        \newline
        Steps:
        \begin{enumerate}
            \item{Student clicks on desired assignment}
            \item{Student can select any of the question(s) in the assignment}
            \item{Student can go back to home page and select any other assignment}
        \end{enumerate}
        Exceptions:
        \begin{itemize}
            \item{There are no assignments available, student has nothing to view.}
        \end{itemize}
    \end{section}
    
    
    \begin{section}{Student Views Question}
        Description: The student views a question that is part of an assignment. \newline
        Primary Actor: Student \newline
        Secondary Actor: System \newline
        Precondition: The student is logged into the system. \newline
        Trigger: The student clicks on the question to view it. \newline
        Successful End Condition: The student views the question they have selected. \newline
        Failed End Condition: The student is unable to view the question. \newline
        \newline
        Steps:
        \begin{enumerate}
            \item{The student selects an assignment to view.}
            \item{The student then selects a specific question to view.}
            \item{The student views the question.}
        \end{enumerate}
        Exceptions: None
    \end{section}

    
    
    \begin{section}{Student Submits Answer to Question}
    
    
    

    
    \begin{section}{Student Views Feedback}
        Description: After submitting the question the student is able to see the feedback that is given for the 
        question based on if they answered the question correctly or not. \newline
        Primary Actor: Student \newline
        Secondary Actor: System \newline
        Precondition: The student has selected a question and answered the question. \newline
        Trigger: The student submits the answer to the question. \newline
        Successful End Condition: Feedback is displayed on the screen for the student 
        to read and use to help make their answer correct if it is incorrect. \newline
        Failed End Condition: There is no feedback that gets displayed. \newline
        \newline
        Steps:
        \begin{enumerate}
            \item{Once the answer has been submitted the system compares it to the correct answer given.}
            \item{If the answer is correct the question page displays the feedback that is 
            written by the author for a correct answer such as good job or correct.}
            \item{If the answer is incorrect the question page displays the feedback that is 
            written by the author for an incorrect answer such as a hint or common mistake.}
            \item{The student reads the feedback that is displayed on the screen.}
        \end{enumerate}
        Exceptions:
        \begin{itemize}
            \item{The author does not enter in any feedback for correct answers so no feedback will be displayed.}
            \item{The author does not enter in any feedback for incorrect answers so no feedback will be displayed.}
        \end{itemize}
    \end{section}
        
    \begin{section}{Author Logs In}
        Description: The author is logging in, in order to access the tutor system using their NetID and password. \newline
        Primary Actor: Author \newline
        Secondary Actor: System \newline
        Precondition: The author is on the login page of the system. \newline
        Trigger: The author wants to log into the tutoring system. \newline
        Successful End Condition: The author successfully logs into the tutoring system. \newline
        Failed End Condition:  The author is unable to log into the system because of an incorrect NetID or password or both \newline
        \newline
        Steps:
        \begin{enumerate}
            \item{The author enters his NetID in the username textbox}
            \item{The author enters his password in the password textbox}
            \item{The author clicks on the LOGIN button to enter the system}
        \end{enumerate}
        Exceptions:
        \begin{itemize}
            \item{If the author does not enter the correct NetID or password 
            the system will not log the student in and a message will appear stating 
            that the NetID or password that they entered is incorrect and to please try again.}
        \end{itemize}
    \end{section}
  
  
    
    \begin{section}{Author Views the Question Bank}
    
    \end{section}
    
    
    
    \begin{section}{Author Adds a New Question to Question Bank}
    
    \end{section}
    
    
    
    \begin{section}{Author Edits a Question From the Question Bank}
        Description: The author wants to edit a question that is already in the question bank by changing any or all of the different parts of the question. \newline
        Primary Actor: Author \newline
        Secondary Actor: System \newline
        Precondition: The author has already created the question that wish to edit. \newline
        Trigger: The author is viewing a question that they wish to edit. \newline
        Successful End Condition: The question changes have been made and the author is prompted that their changes have been made. \newline
        Failed End Condition: The question does not get updated and the author is prompted saying that the question was not changed. \newline
        \newline
        Steps:
        \begin{enumerate}
            \item{The author selects the edit button.}
            \item{The author edits the question text.}
            \item{The author edits the question correct answer.}
            \item{The author edits the question feedback.}
            \item{The author saves the changes to the question and answer.}
        \end{enumerate}
        Exceptions:
        \begin{itemize}
            \item{The author does not want to edit the question text so they do not edit it and leave it alone.}
            \item{The author does not want to edit the question correct answer so they do not edit it and leave it alone.}
            \item{The author does not want to edit the question feedback so they do not edit it and leave it alone.}
            \item{The author does not save the changes to the question so the changes are not saved.}
            \item{The author exits the system without saving the changes to the questions resulting in none of the changes being saved.}
        \end{itemize}
    \end{section}
    
    
    \begin{section}{Author Removes a Question From the Question Bank}
    
    \end{section}
    
    
    \begin{section}{Author Views a Question}
        Description: The author views questions that have been created. \newline
        Primary Actor: Author \newline
        Secondary Actor: System \newline
        Precondition: The author is logged into the system. \newline
        Trigger: The author has created a question and would like to view the question. \newline
        Successful End Condition: The author views the question that they would like to view. \newline
        Failed End Condition: The author is unable to view the question that they want to view. \newline
        \newline
        Steps:
        \begin{enumerate}
            \item{The author selects the questions tab from their view.}
            \item{The author then scrolls through the list of questions.}
            \item{The author then selects the question they would like to view.}
            \item{The author then views the question.}
        \end{enumerate}
        Exceptions: None
    \end{section}
    
    
    \begin{section}{Author Copies a Question} 
    
    \end{section}



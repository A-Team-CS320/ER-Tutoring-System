\chapter{Non-functional Requirements}

\section{Environmental}

\subsection{ENV-01: Game will run on multiple platforms}
\textbf{Description}: The game should run on iOS and Android devices as well
as Facebook through a modern browser.

\subsection{ENV-02: Game will support all screen sizes regardless of platform}
\textbf{Description}: The game will run on all sizes of iOS and Android screens.

\subsection{ENV-03: Game will be in a portrait view with a flexible aspect ratio}
\textbf{Description}: Our game will be displayed in portrait mode only. The
aspect ratio will be exactly the same on all platforms, regardless
of the screen\textquoteright{}s true aspect ratio. When fitting the
display to the device\textquoteright{}s screen, the image will be
scaled, and there may or may not be blank space along the edge of
the screen to accommodate the change in aspect ratio.

\subsection{ENV-04: Game resolution will use vector graphics to scale viewport}
\textbf{Description}: The game will use vector graphics for all of
the graphical components (menus, avatars, sprites, etc.) so that scaling
does not cause pixelation issue on platforms with larger screens. 

\subsection{ENV-05: The user should be able to mute the in-game sound }
\textbf{Description}: The in-game sound should be split up into two different
groups, background music and sound effects. The two should be able
muted/unmuted independently of each other.

\subsection{ENV-06: The user does not need to finish the game in one sitting}
\textbf{Description}: The target user is someone who has a small amount of
free time, for example sitting on a bus. The game should be \textquotedblleft{}addicting\textquotedblright{}
and make the user want to play repeatedly, but individual games should
be fairly short. The user will also be able to pause the game, and
continue at a later time.

\subsection{ENV-07: The Pause button will be displayed on the screen{[}s{]}}
\textbf{Description}: The top-right corner of the screen during gameplay will
always feature the pause button (except when in pause mode), which
can be pressed at any time during gameplay.

\subsection{ENV-08: The menu screens will display the user\textquoteright{}s
current level}
\textbf{Description}: The top-right corner of the screen while the user is
navigating the menus should always say what level the user is currently
at and their progress within that level.

\subsection{ENV-09: The screen should tell the user how much money they have.}
\textbf{Description}: The top-left corner of the screen should always say how
much in-game currency the user has remaining.

\subsection{ENV-10: The store will have a list of categories for the user to
choose. }
\textbf{Description}: Clicking on each of the categories will bring the user
to a scrollable menu featuring all of the relevant items to the category.
The user will then be able to purchase the items. 

\subsection{ENV-11: The user will have access to global and friend-only leaderboards}
\textbf{Description}: The user will be able to access a global leaderboard
or a leaderboard of his/her Facebook friends. There will be a toggle
button to switch between the two views.

\section{Performance}

\subsection{PER-01: Small game file size (<50 megabytes)}
\chapter{Non-functional Requirements}

\section{Environmental}

\subsection{ENV-01: Game will run on multiple platforms}
\textbf{Description}: The game should run on iOS and Android devices as well
as Facebook through a modern browser.

\subsection{ENV-01: Game will support all screen sizes regardless of platform}
\textbf{Description}: The game will run on all sizes of iOS and Android screens.

\subsection{ENV-02: Game will be in a portrait view with a flexible aspect ratio}
\textbf{Description}: Our game will be displayed in portrait mode only. The
aspect ratio will be exactly the same on all platforms, regardless
of the screen\textquoteright{}s true aspect ratio. When fitting the
display to the device\textquoteright{}s screen, the image will be
scaled, and there may or may not be blank space along the edge of
the screen to accommodate the change in aspect ratio.

\subsection{ENV-03: Game resolution will use vector graphics to scale viewport}
\textbf{Description}: The game will use vector graphics for all of
the graphical components (menus, avatars, sprites, etc.) so that scaling
does not cause pixelation issue on platforms with larger screens. 

\subsection{ENV-02: The user should be able to mute the in-game sound }
\textbf{Description}: The in-game sound should be split up into two different
groups, background music and sound effects. The two should be able
muted/unmuted independently of each other.

\subsection{ENV-03: The user does not need to finish the game in one sitting}
\textbf{Description}: The target user is someone who has a small amount of
free time, for example sitting on a bus. The game should be \textquotedblleft{}addicting\textquotedblright{}
and make the user want to play repeatedly, but individual games should
be fairly short. The user will also be able to pause the game, and
continue at a later time.

\subsection{ENV-04: The Pause button will be displayed on the screen{[}s{]}}
\textbf{Description}: The top-right corner of the screen during gameplay will
always feature the pause button (except when in pause mode), which
can be pressed at any time during gameplay.

\subsection{ENV-05: The menu screens will display the user\textquoteright{}s
current level}
\textbf{Description}: The top-right corner of the screen while the user is
navigating the menus should always say what level the user is currently
at and their progress within that level.

\subsection{ENV-06: The screen should tell the user how much money they have.}
\textbf{Description}: The top-left corner of the screen should always say how
much in-game currency the user has remaining.

\subsection{ENV-07: The store will have a list of categories for the user to
choose. }
\textbf{Description}: Clicking on each of the categories will bring the user
to a scrollable menu featuring all of the relevant items to the category.
The user will then be able to purchase the items. 

\subsection{ENV-08: The user will have access to global and friend-only leaderboards}
\textbf{Description}: The user will be able to access a global leaderboard
or a leaderboard of his/her Facebook friends. There will be a toggle
button to switch between the two views.

\section{Performance}

\subsection{PER-01: Small game file size (<50 megabytes)}
\textbf{Description}: As there is a fairly limited amount of content in Line
Bounce, and we have decided not to implement a level-based game structure,
we should be able to keep our game size down to under 50 MB. Over
a 3G connection speed of 200 KB/s, the game file should not take any
longer than 3 minutes to download. Over a 4G connection speed of 1
MB/s, the game file wouldn\textquoteright{}t take any more than a
minute to download. All Wi-Fi connections are different, so the two
above listed download rates should be used as benchmarks to gauge
how quickly the game file can be downloaded

\subsection{PER-02: Non-maximum CPU usage}
\textbf{Description}: Our application should occupy no more than 80\% of the
mobile device\textquoteright{}s CPU during gameplay for rendering
the game. 

\subsection{PER-03: Game frame rate of at least 30 frames per second}
\textbf{Description}: Game will run at 30 frames per second to ensure smoothness
of animation and game speed.

\subsection{PER-04: Fast menu response time}
\textbf{Description}: For the menu, response time should be reasonably fast
so as not to infuriate users, so a 0.01-0.25 second window of lag
time will be implemented.

\subsection{PER-05: Fast in-game response time}
\textbf{Description}: The actual gameplay, we must provide a responsiveness
that appears to be near-instantaneous to the human eye. The amount of lag time
should be less than 24-33 milliseconds for each action. This includes
time will be implemented.

\subsection{PER-05: Fast in-game response time}
\textbf{Description}: The actual gameplay, we must provide a responsiveness
that appears to be near-instantaneous to the human eye. The amount of lag time
should be less than 33 milliseconds for each action. This includes
the lag time between the moment a user presses the screen to draw
a line and when it appears on screen. It also includes the time between
the obtaining of a power-up and the assimilation of its effects.

\subsection{PER-06: Fast database access}
\textbf{Description}: When accessing the store, and when logging in, the delay
time will remain under 5 seconds. The user\textquoteright{}s device
must access the server, which accesses the stored data in the Database,
and then the data is sent back to the user. So the speed of both the
user accessing the server and the server communicating with the DB
to under 2 seconds each.

\section{Accuracy}

\subsection{ACC-01: Input response should be accurate}
\textbf{Description}: The game should respond accurately, where touch and mouse
input should respond in the correct location on the screen. This will
be with an accuracy of +/- 2 pixels.{[}e{]}

\subsection{ACC-02: Straight line approximation of path that player draws}
\textbf{Description}: The player draws a platform for the avatar to bounce
from by touching and dragging across the screen, and then lifting
their finger to mark the end of the platform. The accuracy of this
is reliant on the API used to gather the touch information, which
will differ between iOS and Android as well as among the individual
devices using these platforms. On Facebook this accuracy is inherent,
as the mouse is a much more exact means of drawing than a finger.
The first touch and lifting of the finger will mark the start and
end of the line. For the Facebook game, the left click, drag and release
of the left click will mark the start and end of the line respectively.

\subsection{ACC-03: Accidental touches will be ignored by the game engine}
\textbf{Description}: The game will ignore taps (without finger dragging) in
the main, gameplay portion of the screen, as only swipes draw platforms
in game and this prevents accidental touches. The user will have to
drag their finger a distance at least equal to the width of the avatar
in order for the line to register. The pause button, however, will
be activated by single-point touches, and not by swipes.

\subsection{ACC-04: People with larger fingers will still be able to play the game}
\textbf{Description}: For people with larger fingers their line will start
drawing where the middle of their finger touches the screen , allowing
even large fingers to draw lines with a reasonable degree of accuracy.
When playing the game on Facebook this is not an issue, as the mouse
is used to draw lines. 

\subsection{ACC-05: In an optimal game the player climbs forever}
\textbf{Description}: While an optimal game is not technically possible it
allows us to set a standard. A good game would be one where the player
reaches a new high score for their avatar\textquoteright{}s height,
reaching a higher point on the map than previously achieved. The map
will auto-generate as the avatar climbs. At least the height of two
screens will be generated past the top of the screen in preparation
for the avatar to climb.

\subsection{ACC-06: Avatar will bounce upon touching the platform}
\textbf{Description}: The avatar should bounce off of a platform as long as
the avatar touches the platform and they have downward velocity.

\subsection{ACC-07: Bounce acceleration calculation has at least 3 significant figures}
\textbf{Description}: When the avatar collides with a platform the acceleration
due to this bounce must be correctly calculated to within at least
6 significant figures. Using linear algebraic mathematical formulas,
the game will calculate the angle that the avatar collided with a
platform and find the reflection vector. That information is then
used to apply an impulse to the avatar in the appropriate direction
with an accurate acceleration based on the size of the line as stated
in FUN-23.

\subsection{ACC-08: Avatar will die upon touching any portion of an enemy}
\textbf{Description}: As long as a portion of the user\textquoteright{}s sprite
touches touches an enemy, the user will either die or lose their shield
if they have a shield powerup equipped.

\section{Robustness}

\subsection{ROB-01: Game should never crash the user\textquoteright{}s device}
\textbf{Description}: The game will never cause a fatal crash of the user\textquoteright{}s
device. The game may crash but should never crash the device itself.

\subsection{ROB-02: Buttons should always perform desired functionality}
\textbf{Description}: The buttons should be straightforward in what they do
and the user should never have to question what a button does.

\subsection{ROB-03: The game must never corrupt any database}
\textbf{Description}: Game player status and payment records must never be
lost, especially after crash or loss of connection. No queries sent
to the database should corrupt any tables.

\subsection{ROB-04: Purchases should be easily recovered}
\textbf{Description}: A recovery function linked to the account of the player
will be implemented. This will allow the user to sync any data to
their platform if something is to go wrong with the auto-sync functionality.
It is accessible in the Settings menu.

\subsection{ROB-05: The game will be easily extensible}
\textbf{Description}: It will be easy to add new features to the game. This
may include game modes, power-ups, enemy types, store categories and
avatar customization.

\section{Safety}

\subsection{SAF-01: Game should be family friendly (G-rated)}
\textbf{Description}: Game will be made suitable to play for all audiences.

\subsection{SAF-02: Warn users of extended play sessions}
\textbf{Description}: The game will warn users who have been playing for a
long continuous period of time that they should take a break (\textquotedblleft{}It
appears you\textquoteright{}ve been playing for a while, please take
a break\textquotedblright{}). However, the game will never force users
to exit. This will happen after two hours of continuous play.

\subsection{SAF-03: Bugs should be easily patchable}
\textbf{Description}: The application should be written in a clear, concise
manner and thoroughly abstracted and DRYed (Don\textquoteright{}t
Repeat Yourself) out so that writing patches for bugs is simple and
easy. A single code-base ensures that one patch will fix the bug for
every platform

\subsection{SAF-04: Game should keep track of in-game purchases}
\textbf{Description}: Local and remote purchase tracking required. This data
will be automatically synced if needed (e.g. The user plays a game
on a different platform than the one they made purchases on). 

\subsection{SAF-05: Confirm transaction details}
\textbf{Description}: Before processing transaction, ask customer to confirm
the transaction details. A small popup will be displayed on the current
page with the item the user is purchasing as well as the price they
are agreeing to pay.

\subsection{SAF-06: Game will include a fail-soft database}
\textbf{Description}: Keep database information minimal so that the game does
not have to connect to the database very often. There will also be
a second database that will be synced up regularly with the original
in case of a failure.

\subsection{SAF-07: Backup database regularly}
\textbf{Description}: The database will be backed up every 15 minutes to minimize
lost customer purchases.

\section{Security}

\subsection{SEC-01: Game will use Facebook authentication}
\textbf{Description}: Facebook OAuth will be used for authentication. This
allows the user to log in to their Facebook account securely. The
game will need to use OAuth correctly for the game to be secure. Facebook
provides information on correct implementation.

\subsection{SEC-02: Game will use Facebook payment system}
\textbf{Description}: Facebook\textquoteright{}s API for secure payment transactions
will be used, as requested by the client. This is the only method
of payment accepted. The game will need to use the Facebook payment
system correctly for the game to be secure. Facebook provides information
on correct implementation. Transactions will not be lost, as specified
in ROB-02.

\subsection{SEC-03: Data will be stored on a secure database server}
\textbf{Description}: User purchase history and scores will be stored on the
server. The database will only be accessed by the game server, not
the game clients. This allows for only the game server to hold the
credentials to the database. 

\subsection{SEC-04: Data storage for both client and server}
\textbf{Description}: Server-side data will be used as temporary storage until
the secure database is updated with the user\textquoteright{}s progress.
It will push the changes to the database when the user logs out of
Facebook or a connection is lost. Client-side data will be for the
user to play in offline mode if he/she does not want to utilize these
features or if they does not currently have an active internet connection.

\subsection{SEC-05: Data will be sanitized}
\textbf{Description}: Data sanitization will protect against attacks by malicious
users. For all input boxes, the use of HTML tags, JavaScript or any
other possible code that could jeopardize the security of the game
\textbf{Description}: As there is a fairly limited amount of content in Line
Bounce, and we have decided not to implement a level-based game structure,
we should be able to keep our game size down to under 50 MB. Over
a 3G connection speed of 200 KB/s, the game file should not take any
longer than 3 minutes to download. Over a 4G connection speed of 1
MB/s, the game file wouldn\textquoteright{}t take any more than a
minute to download. All Wi-Fi connections are different, so the two
above listed download rates should be used as benchmarks to gauge
how quickly the game file can be downloaded

\subsection{PER-02: Non-maximum CPU usage}
\textbf{Description}: Our application should occupy no more than 80\% of the
mobile device\textquoteright{}s CPU during gameplay for rendering
the game. 

\subsection{PER-03: Game frame rate of at least 30 frames per second}
\textbf{Description}: Game will run at 30 frames per second to ensure smoothness
of animation and game speed.

\subsection{PER-04: Fast menu response time}
\textbf{Description}: For the menu, response time should be reasonably fast
so as not to infuriate users, so a 0.01-0.25 second window of lag
time will be implemented.

\subsection{PER-05: Fast in-game response time}
\textbf{Description}: The actual gameplay, we must provide a responsiveness
that appears to be near-instantaneous to the human eye. The amount of lag time
should be less than 24-33 milliseconds for each action. This includes
time will be implemented.

\subsection{PER-05: Fast in-game response time}
\textbf{Description}: The actual gameplay, we must provide a responsiveness
that appears to be near-instantaneous to the human eye. The amount of lag time
should be less than 33 milliseconds for each action. This includes
the lag time between the moment a user presses the screen to draw
a line and when it appears on screen. It also includes the time between
the obtaining of a power-up and the assimilation of its effects.

\subsection{PER-06: Fast database access}
\textbf{Description}: When accessing the store, and when logging in, the delay
time will remain under 5 seconds. The user\textquoteright{}s device
must access the server, which accesses the stored data in the Database,
and then the data is sent back to the user. So the speed of both the
user accessing the server and the server communicating with the DB
to under 2 seconds each.

\section{Accuracy}

\subsection{ACC-01: Input response should be accurate}
\textbf{Description}: The game should respond accurately, where touch and mouse
input should respond in the correct location on the screen. This will
be with an accuracy of +/- 2 pixels.{[}e{]}

\subsection{ACC-02: Straight line approximation of path that player draws}
\textbf{Description}: The player draws a platform for the avatar to bounce
from by touching and dragging across the screen, and then lifting
their finger to mark the end of the platform. The accuracy of this
is reliant on the API used to gather the touch information, which
will differ between iOS and Android as well as among the individual
devices using these platforms. On Facebook this accuracy is inherent,
as the mouse is a much more exact means of drawing than a finger.
The first touch and lifting of the finger will mark the start and
end of the line. For the Facebook game, the left click, drag and release
of the left click will mark the start and end of the line respectively.

\subsection{ACC-03: Accidental touches will be ignored by the game engine}
\textbf{Description}: The game will ignore taps (without finger dragging) in
the main, gameplay portion of the screen, as only swipes draw platforms
in game and this prevents accidental touches. The user will have to
drag their finger a distance at least equal to the width of the avatar
in order for the line to register. The pause button, however, will
be activated by single-point touches, and not by swipes.

\subsection{ACC-04: People with larger fingers will still be able to play the game}
\textbf{Description}: For people with larger fingers their line will start
drawing where the middle of their finger touches the screen , allowing
even large fingers to draw lines with a reasonable degree of accuracy.
When playing the game on Facebook this is not an issue, as the mouse
is used to draw lines. 

\subsection{ACC-05: In an optimal game the player climbs forever}
\textbf{Description}: While an optimal game is not technically possible it
allows us to set a standard. A good game would be one where the player
reaches a new high score for their avatar\textquoteright{}s height,
reaching a higher point on the map than previously achieved. The map
will auto-generate as the avatar climbs. At least the height of two
screens will be generated past the top of the screen in preparation
for the avatar to climb.

\subsection{ACC-06: Avatar will bounce upon touching the platform}
\textbf{Description}: The avatar should bounce off of a platform as long as
the avatar touches the platform and they have downward velocity.

\subsection{ACC-07: Bounce acceleration calculation has at least 3 significant figures}
\textbf{Description}: When the avatar collides with a platform the acceleration
due to this bounce must be correctly calculated to within at least
6 significant figures. Using linear algebraic mathematical formulas,
the game will calculate the angle that the avatar collided with a
platform and find the reflection vector. That information is then
used to apply an impulse to the avatar in the appropriate direction
with an accurate acceleration based on the size of the line as stated
in FUN-23.

\subsection{ACC-08: Avatar will die upon touching any portion of an enemy}
\textbf{Description}: As long as a portion of the user\textquoteright{}s sprite
touches touches an enemy, the user will either die or lose their shield
if they have a shield powerup equipped.

\section{Robustness}

\subsection{ROB-01: Game should never crash the user\textquoteright{}s device}
\textbf{Description}: The game will never cause a fatal crash of the user\textquoteright{}s
device. The game may crash but should never crash the device itself.

\subsection{ROB-02: Buttons should always perform desired functionality}
\textbf{Description}: The buttons should be straightforward in what they do
and the user should never have to question what a button does.

\subsection{ROB-03: The game must never corrupt any database}
\textbf{Description}: Game player status and payment records must never be
lost, especially after crash or loss of connection. No queries sent
to the database should corrupt any tables.

\subsection{ROB-04: Purchases should be easily recovered}
\textbf{Description}: A recovery function linked to the account of the player
will be implemented. This will allow the user to sync any data to
their platform if something is to go wrong with the auto-sync functionality.
It is accessible in the Settings menu.

\subsection{ROB-05: The game will be easily extensible}
\textbf{Description}: It will be easy to add new features to the game. This
may include game modes, power-ups, enemy types, store categories and
avatar customization.

\section{Safety}

\subsection{SAF-01: Game should be family friendly (G-rated)}
\textbf{Description}: Game will be made suitable to play for all audiences.

\subsection{SAF-02: Warn users of extended play sessions}
\textbf{Description}: The game will warn users who have been playing for a
long continuous period of time that they should take a break (\textquotedblleft{}It
appears you\textquoteright{}ve been playing for a while, please take
a break\textquotedblright{}). However, the game will never force users
to exit. This will happen after two hours of continuous play.

\subsection{SAF-03: Bugs should be easily patchable}
\textbf{Description}: The application should be written in a clear, concise
manner and thoroughly abstracted and DRYed (Don\textquoteright{}t
Repeat Yourself) out so that writing patches for bugs is simple and
easy. A single code-base ensures that one patch will fix the bug for
every platform

\subsection{SAF-04: Game should keep track of in-game purchases}
\textbf{Description}: Local and remote purchase tracking required. This data
will be automatically synced if needed (e.g. The user plays a game
on a different platform than the one they made purchases on). 

\subsection{SAF-05: Confirm transaction details}
\textbf{Description}: Before processing transaction, ask customer to confirm
the transaction details. A small popup will be displayed on the current
page with the item the user is purchasing as well as the price they
are agreeing to pay.

\subsection{SAF-06: Game will include a fail-soft database}
\textbf{Description}: Keep database information minimal so that the game does
not have to connect to the database very often. There will also be
a second database that will be synced up regularly with the original
in case of a failure.

\subsection{SAF-07: Backup database regularly}
\textbf{Description}: The database will be backed up every 15 minutes to minimize
lost customer purchases.

\section{Security}

\subsection{SEC-01: Game will use Facebook authentication}
\textbf{Description}: Facebook OAuth will be used for authentication. This
allows the user to log in to their Facebook account securely. The
game will need to use OAuth correctly for the game to be secure. Facebook
provides information on correct implementation.

\subsection{SEC-02: Game will use Facebook payment system}
\textbf{Description}: Facebook\textquoteright{}s API for secure payment transactions
will be used, as requested by the client. This is the only method
of payment accepted. The game will need to use the Facebook payment
system correctly for the game to be secure. Facebook provides information
on correct implementation. Transactions will not be lost, as specified
in ROB-02.

\subsection{SEC-03: Data will be stored on a secure database server}
\textbf{Description}: User purchase history and scores will be stored on the
server. The database will only be accessed by the game server, not
the game clients. This allows for only the game server to hold the
credentials to the database. 

\subsection{SEC-04: Data storage for both client and server}
\textbf{Description}: Server-side data will be used as temporary storage until
the secure database is updated with the user\textquoteright{}s progress.
It will push the changes to the database when the user logs out of
Facebook or a connection is lost. Client-side data will be for the
user to play in offline mode if he/she does not want to utilize these
features or if they does not currently have an active internet connection.

\subsection{SEC-05: Data will be sanitized}
\textbf{Description}: Data sanitization will protect against attacks by malicious
users. For all input boxes, the use of HTML tags, JavaScript or any
other possible code that could jeopardize the security of the game

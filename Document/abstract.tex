%\chapter{Introduction}
	\begin{chapter}{Purpose of System}
		The purpose of this system is to be an online tutor for students learning ER diagrams. 
		This learning tool would help teach the students and allow them to create ER diagrams and submit 
		them as assignments. The assignments will be created by the instructor using questions taken from 
		a question bank. The questions will have been created by an author and will include the correct answers 
		and any necessary feedback. Once the student submits an answer to be graded they will be given feedback 
		if incorrect. This helps the students learn from their mistakes. The benefits of this system is that it 
		makes learning and submitting assignments much easier for both the students and instructor.
	\end{chapter}

	\begin{chapter}{Overview}
		This document will clearly outline the requirements and the system requirements of the ER Diagram 
		Tutoring System. This document should cover all key concepts and fundamental requirements. 
		The requirements should satisfy the standards of the client and provide accuracy in the functionality 
		of the system.
	\end{chapter}

	\begin{chapter}{Scope}
		The ER Diagram Tutoring System will contain many features to help both students and instructors with 
		the process of teaching. To use the system a user will have to log in. There will be three different 
		types of users. An author can create, edit, copy, or remove questions from the question bank. 
		They are also able to view any previously made question without changing it. An instructor can 
		make assignments using questions taken from the question bank. These assignments are to be assigned 
		to students in the class. They are also able to view their students and information about their 
		attempts on the assignments. The students are the ones who are taking the class and who the questions 
		are made for. They can view questions, without the answers shown, and need to answer and submit them. 
	\end{chapter}

	\begin{chapter}{Definitions}
 		\begin{itemize}
 			\item{\textbf{Online Tutor System}  the system that is created which is a web based platform, 
 			used to assign students questions and assignments and used by students to learn how to 
 			construct ER Diagrams}
			\item{\textbf{Student}  the user that is accessing the online tutor system in order to learn 
			and practice making ER Diagrams}
			\item{\textbf{Learning Tool} - the system which is used by the student to understand how to 
			create ER Diagrams}
			\item{\textbf{Assignment} - the collection of questions that the instructor selects for the 
			student to answer by a certain common date}
			\item{\textbf{Instructor} - the user who creates assignments by selecting questions from the 
			question bank to be assigned to their students to answer}
			\item{\textbf{Feedback} - what is returned to students after they submit an answer.  It can be 
			composed of hints, suggestions and messages from the author}
			\item{\textbf{Author} - the user who creates the questions, answer, and feedback.  They also 
			upload them to the question bank for future use}
			\item{\textbf{Draw} - the action the student or author takes to insert shapes and words from 
			the toolbox to the answer area of the UI, either to be submitted by the student, or while 
			being created by the author}
			\item{\textbf{Question} - what is being asked of the student to create, part of an assignment}
			\item{\textbf{Toolbox} - the area of the UI that contains all of the shapes, lines and textboxes 
			for the user to drag and drop onto the draw space}
			\item{\textbf{Answer} - the ER Diagram that is submitted to be graded}
			\item{\textbf{NetID/username} - the NetID is the username of the user needed to log into the 
			system}
			\item{\textbf{Question bank} - database of all of the questions that have been created by an 
			author that the instructors can select from to create assignments for the students}
			\item{\textbf{Draw space} - the area in which the answer to the question is drawn}
			\item{\textbf{Users} - the person using the system. There are three different types of users: 
			Students, Instructors, and Authors}
		\end{itemize}
	\end{chapter}

\chapter {System Overview}

The purpose of this system is to be an online tutor for students who are learning to create ER diagrams. This learning tool will teaches students to build ER diagrams and allows them to submit the diagrams that they create as responses to assignment questions. The assignments are be created by the instructor using questions that are taken from the question bank. These questions are created by the author and they include the correct answers and any necessary feedback. Once the student submits an answer to be graded they will be given feedback if incorrect. This helps the students learn from their mistakes. The benefits of this system are that it makes learning and submitting assignments much easier for both the students and instructor.

\chapter{Stakeholders}
The stakeholders of the system involve the three groups defined as users, as well as the client, management, and the developers. The administrators of the system will be involved in managing the other two user groups authors and 
The stakeholders of the system involve three layers of content interactors, and the administrators adopting the system. The administrators of the system will be involved in managing the users that are allowed direct access and ensuring that only the professors who have use of an ER diagram learning system have access as well. While the system itself has the capability to manage direct user registration the administrators have a stake in ensuring that the system is accessible to those who need it. The content interactors all have a stake in either accessing, modifying, or creating the questions hosted within the system, and the question bank in general. The professors who have access to the system have a stake in creating groups of questions for students and in a lesser sense creating or editing questions to their needs, and accessing reported student progress on those content clusters. Content creators primarily add and edit questions in the bank and will not directly interface with student answers. Students do not create or edit questions, but answer and submit content clusters as assigned by professors. 

\chapter{Scope of the Document}
The online tutoring system will contain many features to help both students and instructors with the process of teaching. To use the system a user will have to log in. There will be three different types of users. An author can create, edit, copy, or remove questions from the question bank. They are also able to view any previously made question without changing it. An instructor can make assignments using questions taken from the question bank. These assignments are to be assigned to students in the class. They are also able to view their students and information about their attempts on the assignments. The students are the ones who are taking the class and who the questions are made for. They can view questions, without the answers shown, and need to answer and submit them.

\chapter{Definitions}
TODO